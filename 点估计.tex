\chapter{点估计}

\section{点估计法}

\subsection{矩估计法}

样本原点矩是原点矩的无偏估计.

样本中心矩是中心矩的渐进无偏估计.

\subsection{极大似然估计法}

若存在充分统计量, 则极大似然估计是充分统计量的函数.

$L(\theta;\vec{x})$ 的极大似然估计和 $c(\vec{x})L(\theta;\vec{x})$ 的极大似然估计相同.

$L(\theta;\vec{x})$ 的极大似然估计和 $\ln L(\theta;\vec{x})$ 的极大似然估计相同.

不变原理: 若 $\theta$ 的极大似然估计为 $\hat{\theta}_{\text{MLE}}$, 则 $g(\theta)$ 的极大似然估计为 $g(\hat{\theta}_{\text{MLE}})$.

\section{点估计评优}

\section{UMVUE}

两点, Poisson, $\gamma$, Gaussian 分布的参数的常用估计是UMVUE.
